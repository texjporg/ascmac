% \iffalse meta-comment
%
%  Copyright (c) 2018 Japanese TeX Development Community
%
%  This file is part of ascmac community edition.
%  -------------------------------------------------------------
%
% \fi
%
%
% \iffalse
%
% 2018/02/07 v0.3 (HY)
%     - First public version.
%
% \fi
%
% \iffalse
%<*driver>
\def\eTeX{$\varepsilon$-\pTeX}
\def\pTeX{p\kern-.15em\TeX}
\def\upTeX{u\pTeX}
\def\epTeX{$\varepsilon$-\pTeX}
\def\eupTeX{$\varepsilon$-\upTeX}
\def\pLaTeX{p\kern-.05em\LaTeX}
\def\pLaTeXe{p\kern-.05em\LaTeXe}
\def\upLaTeX{u\pLaTeX}
\def\upLaTeXe{u\pLaTeXe}
%</driver>
% \fi
%
% \iffalse
%<*varascmac>
%% varascmac.sty
%% written by Hironobu Yamashita (@aminophen)
%</varascmac>
%<varascmac>\NeedsTeXFormat{LaTeX2e}
%<*driver>
\NeedsTeXFormat{pLaTeX2e}
\ProvidesFile{varascmac.dtx}
%</driver>
%<varascmac>\ProvidesPackage{varascmac}
  [2018/02/07 v0.3 ascmac variant]
%<*driver>
\documentclass{jltxdoc}
\usepackage{amssymb}
\usepackage[usedtou]{varascmac}
\GetFileInfo{varascmac.dtx}
\title{The \textsf{varascmac} package \fileversion}
\author{Japanese \TeX\ Development Community}
\date{作成日:\filedate}
\begin{document}
  \maketitle
  \DocInput{\filename}
\end{document}
%</driver>
% \fi
%
% \tableofcontents
%
% \section{はじめに}
%
% この\textsf{varascmac}パッケージは、
% \textsf{ascmac}パッケージの変形版かつ追補版です。
% 既定で\textsf{graphics}パッケージを読み込んで利用します。
% ただし、\pLaTeX{}/\upLaTeX{}とdvipdfmxの組み合わせの場合は、
% \verb+[usedtou]+オプションを指定することで
% \textsf{graphics}不要にもできます
% (\pTeX{}のプリミティブだけで同等の処理を実現可能のため)。
%
% \section{使いかた}
%
% \subsection{環境型の命令}
%
% \DescribeEnv{varboxnote}
% オリジナルの |boxnote| 環境の変形版である
% |varboxnote| 環境を作りました。
%
% \begin{minipage}{.45\linewidth}\begin{boxnote}
% オリジナルの |boxnote| 環境です。
% ノートを破ったような線は、横組では「上」に、
% 縦組では「右」に現れます。
% \end{boxnote}\end{minipage}\hfill
% \begin{minipage}{.45\linewidth}\begin{varboxnote}
% 新しい |varboxnote| 環境です。
% ノートを破ったような線は、横組では「左」に、
% 縦組では「上」に現れます。
% \end{varboxnote}\end{minipage}
%
% \subsection{その他の命令}
%
% \DescribeMacro{\asciitriangleright}
% \DescribeMacro{\asciitriangleleft}
% \DescribeMacro{\asciitriangleup}
% \DescribeMacro{\asciitriangledown}
% 付属のascgrpフォントに含まれる{\return}記号などは、
% オリジナル版の\textsf{ascmac}パッケージで提供されていましたが、
% 三角形の記号はパッケージ化されていませんでしたので、追加します。
% それぞれ
% \asciitriangleright\asciitriangleleft
% \asciitriangleup\asciitriangledown
% となります。
%
% この機能は\textsf{amssymb}パッケージが提供する
% $\blacktriangleright \blacktriangleleft
%  \blacktriangle \blacktriangledown$
% (|\blacktriangleright|, |\blacktriangleleft|,
% |\blacktriangle|, |\blacktriangledown|)
% に似ていますが、微妙に形状が違います。
% また、\textsf{varascmac}パッケージのものは数式モードで
% なくても利用できます。
%
% \StopEventually{}
%
% \section{コード}
%
% 本パッケージはオリジナルの\textsf{ascmac}パッケージの追補版
% なので、これを先に読み込みます。
%    \begin{macrocode}
%<*varascmac>
\RequirePackage{ascmac}
%    \end{macrocode}
%
% \begin{macro}{\asciitriangleright}
% \begin{macro}{\asciitriangleleft}
% \begin{macro}{\asciitriangleup}
% \begin{macro}{\asciitriangledown}
% 三角形です。
%    \begin{macrocode}
%%
%% Triangles using ascgrp font
\newcommand{\asciitriangleright}{{\@ascgrp 0}}
\newcommand{\asciitriangleleft}{{\@ascgrp 1}}
\newcommand{\asciitriangleup}{{\@ascgrp 2}}
\newcommand{\asciitriangledown}{{\@ascgrp 3}}
%    \end{macrocode}
% \end{macro}
% \end{macro}
% \end{macro}
% \end{macro}
%
% パッケージオプションを定義し、ユーザ指定に従い実行します。
% |usedtou| というオプションが利用可能です。(何をするのかは
% この後の |varboxnote| 環境の実装部分の説明を参照。)
%    \begin{macrocode}
%%
\def\vascmac@status{0}
\DeclareOption{usedtou}{\def\vascmac@status{1}}
\ProcessOptions\relax
%%
%    \end{macrocode}
%
% |usedtou| オプションは\pTeX{}/\upTeX{}以外では利用不可のため、
% 警告を出してデフォルトにフォールバックします。
%    \begin{macrocode}
%% On pdfLaTeX etc, `usedtou' falls back to graphics rotation
\if 1\vascmac@status
  \ifascmac@ptex\else
    \PackageWarning{varascmac}{%
      Option `usedtou' requires pLaTeX, upLaTeX\MessageBreak
      or LuaLaTeX with LuaTeX-ja support!\MessageBreak
      I'm going to ignore it ...}
    \def\vascmac@status{0}
  \fi
\fi
%%
%    \end{macrocode}
%
% (u)p\LaTeXe{}以外をサポートするためのトリックです。
% 「Q」という文字のカテゴリーコードをこのパッケージを読んでいる間だけ
% 変更し、(u)p\LaTeXe{}では無視する文字に、それ以外では
% コメント文字(|%|と同じ)にします。
% ただし、依存パッケージ(ここでは\textsf{graphics})は
% その前に読み込まなければなりません。
%    \begin{macrocode}
%% Required package should be loaded before \catcode trick
\if 0\vascmac@status
  \RequirePackage{graphics}
\fi
%%
\chardef\ascmac@q@catcode=\catcode`\Q\relax
\ifascmac@ptex
  \catcode`\Q=9\relax
\else
  \catcode`\Q=14\relax
\fi
%%
%    \end{macrocode}
%
% ここからは(u)\pTeX{}以外の場合、あるいは
% \verb+usedtou+オプションが無効な場合のコードです。
%    \begin{macrocode}
\if 0\vascmac@status
%    \end{macrocode}
%
% \begin{environment}{varboxnote}
% |varboxnote| 環境を、|\rotatebox| マクロを使用して定義します。
%    \begin{macrocode}
%% Common implementation: requires graphics package
%    \end{macrocode}
%
%    \begin{macrocode}
\def\varboxnote{\par\vspace{.3\baselineskip}%
Q \@saveybaselineshift\ybaselineshift\ybaselineshift\z@
Q \@savetbaselineshift\tbaselineshift\tbaselineshift\z@
  \@bw=\linewidth \advance\@bw-42.16pt
  \setbox\@nbody=\hbox\bgroup\begin{minipage}{\@bw}% (KN:1998/02/27)
Q   \ybaselineshift\@saveybaselineshift \tbaselineshift\@savetbaselineshift
}%
%    \end{macrocode}
%
%    \begin{macrocode}
\def\endvarboxnote{\end{minipage}\egroup
  \@tempdima=\ht\@nbody
  \advance\@tempdima\dp\@nbody \advance\@tempdima40pt
  \setbox\@nbox=\hbox{\@ascgrp b\hss a\hss b}%
  \@whiledim \wd\@nbox<\@tempdima \do{%
    \setbox\@nbox=\hbox{\@ascgrp\unhbox\@nbox \hss a\hss b}}%
  \setbox\@nbox=\hb@xt@\@tempdima{\@ascgrp\unhbox\@nbox \hss a\hss b}%
  \vbox{\parindent\z@\offinterlineskip
  \hrule\@height1.08pt
  \hb@xt@\hsize{\rotatebox{90}{\raise\dp\@nbox\box\@nbox}\hfil\vbox{\vskip20pt
  \box\@nbody\vskip20pt}\hfil\vrule\@width1.08pt}%
  \hrule\@height1.08pt}}
%%
%    \end{macrocode}
% \end{environment}
%
% トリックに使用した「Q」という文字のカテゴリーコードを元に戻して
% 読込を終了します。
%    \begin{macrocode}
\catcode`\Q=\ascmac@q@catcode\relax
\expandafter\endinput
\fi
%%
%    \end{macrocode}
%
% ここからは(u)\pTeX{}専用、かつ\verb+usedtou+オプションが
% 有効な場合のコードです。
% \pTeX{}の場合、\textsf{graphics}パッケージに頼らなくても、
% |varboxnote|環境の描画を実現することができます。
% 具体的には、組方向変更プリミティブである |\dtou| や
% |\yoko| を使用します。
% ただし、|\dtou| 組方向をサポートしているDVIドライバ
% は少なく\footnote{\TeX\ Live 2017時点では、dvipdfmxしか
% 知られていません。dvipsも\cs{dtou}をサポートしていません。}、
% \pLaTeX{}や\upLaTeX{}で作成したDVIをdvipdfmxで処理しなければ
% ならないという制約が生じるため、
% 隠しオプションのような扱いにしておきます。
%
%    \begin{macrocode}
\if 1\vascmac@status
%    \end{macrocode}
%
% \begin{environment}{varboxnote}
% |varboxnote| 環境を、|\dtou| プリミティブを使用して定義します。
%    \begin{macrocode}
%% Another implementation: uses \iftdir, \yoko and \dtou
\def\asc@dtou@dir{\iftdir\yoko\else\dtou\fi}
%    \end{macrocode}
%
%    \begin{macrocode}
\def\varboxnote{\par\vspace{.3\baselineskip}%
  \@saveybaselineshift\ybaselineshift\ybaselineshift\z@
  \@savetbaselineshift\tbaselineshift\tbaselineshift\z@
  \@bw=\linewidth \advance\@bw-42.16pt
  \setbox\@nbody=\hbox\bgroup\begin{minipage}{\@bw}% (KN:1998/02/27)
    \ybaselineshift\@saveybaselineshift \tbaselineshift\@savetbaselineshift
}%
%    \end{macrocode}
%
%    \begin{macrocode}
\def\endvarboxnote{\end{minipage}\egroup
  \@tempdima=\ht\@nbody \advance\@tempdima20pt
  \iftdir\else \advance\@tempdima\dp\@nbody \advance\@tempdima20pt \fi
  \setbox\@nbox=\hbox{\asc@dtou@dir\@ascgrp b\hss a\hss b}%
  \@whiledim \ht\@nbox<\@tempdima \do{%
    \setbox\@nbox=\hbox{\asc@dtou@dir\@ascgrp\unhbox\@nbox \hss a\hss b}}%
  \iftdir \advance\@tempdima\dp\@nbody \advance\@tempdima20pt \fi
  \setbox\@nbox=\hb@xt@\@tempdima{\asc@dtou@dir\@ascgrp\unhbox\@nbox \hss a\hss b}%
  \vbox{\parindent\z@\offinterlineskip
  \hrule\@height1.08pt
  \hb@xt@\hsize{\raise\dp\@nbox\box\@nbox\hfil\vbox{\vskip20pt
  \box\@nbody\vskip20pt}\hfil\vrule\@width1.08pt}%
  \hrule\@height1.08pt}}
%%
%    \end{macrocode}
% \end{environment}
%
% トリックに使用した「Q」という文字のカテゴリーコードを元に戻して
% 読込を終了します。
%    \begin{macrocode}
\catcode`\Q=\ascmac@q@catcode\relax
\expandafter\endinput
\fi
%%
%</varascmac>
%    \end{macrocode}
%
% \Finale
%
\endinput
